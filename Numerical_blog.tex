\documentclass[]{article}
\usepackage{lmodern}
\usepackage{amssymb,amsmath}
\usepackage{ifxetex,ifluatex}
\usepackage{fixltx2e} % provides \textsubscript
\ifnum 0\ifxetex 1\fi\ifluatex 1\fi=0 % if pdftex
  \usepackage[T1]{fontenc}
  \usepackage[utf8]{inputenc}
\else % if luatex or xelatex
  \ifxetex
    \usepackage{mathspec}
  \else
    \usepackage{fontspec}
  \fi
  \defaultfontfeatures{Ligatures=TeX,Scale=MatchLowercase}
\fi
% use upquote if available, for straight quotes in verbatim environments
\IfFileExists{upquote.sty}{\usepackage{upquote}}{}
% use microtype if available
\IfFileExists{microtype.sty}{%
\usepackage{microtype}
\UseMicrotypeSet[protrusion]{basicmath} % disable protrusion for tt fonts
}{}
\usepackage[margin=1in]{geometry}
\usepackage{hyperref}
\hypersetup{unicode=true,
            pdftitle={ChebyChev Analysis},
            pdfauthor={Kory Illenye},
            pdfborder={0 0 0},
            breaklinks=true}
\urlstyle{same}  % don't use monospace font for urls
\usepackage{color}
\usepackage{fancyvrb}
\newcommand{\VerbBar}{|}
\newcommand{\VERB}{\Verb[commandchars=\\\{\}]}
\DefineVerbatimEnvironment{Highlighting}{Verbatim}{commandchars=\\\{\}}
% Add ',fontsize=\small' for more characters per line
\usepackage{framed}
\definecolor{shadecolor}{RGB}{248,248,248}
\newenvironment{Shaded}{\begin{snugshade}}{\end{snugshade}}
\newcommand{\KeywordTok}[1]{\textcolor[rgb]{0.13,0.29,0.53}{\textbf{#1}}}
\newcommand{\DataTypeTok}[1]{\textcolor[rgb]{0.13,0.29,0.53}{#1}}
\newcommand{\DecValTok}[1]{\textcolor[rgb]{0.00,0.00,0.81}{#1}}
\newcommand{\BaseNTok}[1]{\textcolor[rgb]{0.00,0.00,0.81}{#1}}
\newcommand{\FloatTok}[1]{\textcolor[rgb]{0.00,0.00,0.81}{#1}}
\newcommand{\ConstantTok}[1]{\textcolor[rgb]{0.00,0.00,0.00}{#1}}
\newcommand{\CharTok}[1]{\textcolor[rgb]{0.31,0.60,0.02}{#1}}
\newcommand{\SpecialCharTok}[1]{\textcolor[rgb]{0.00,0.00,0.00}{#1}}
\newcommand{\StringTok}[1]{\textcolor[rgb]{0.31,0.60,0.02}{#1}}
\newcommand{\VerbatimStringTok}[1]{\textcolor[rgb]{0.31,0.60,0.02}{#1}}
\newcommand{\SpecialStringTok}[1]{\textcolor[rgb]{0.31,0.60,0.02}{#1}}
\newcommand{\ImportTok}[1]{#1}
\newcommand{\CommentTok}[1]{\textcolor[rgb]{0.56,0.35,0.01}{\textit{#1}}}
\newcommand{\DocumentationTok}[1]{\textcolor[rgb]{0.56,0.35,0.01}{\textbf{\textit{#1}}}}
\newcommand{\AnnotationTok}[1]{\textcolor[rgb]{0.56,0.35,0.01}{\textbf{\textit{#1}}}}
\newcommand{\CommentVarTok}[1]{\textcolor[rgb]{0.56,0.35,0.01}{\textbf{\textit{#1}}}}
\newcommand{\OtherTok}[1]{\textcolor[rgb]{0.56,0.35,0.01}{#1}}
\newcommand{\FunctionTok}[1]{\textcolor[rgb]{0.00,0.00,0.00}{#1}}
\newcommand{\VariableTok}[1]{\textcolor[rgb]{0.00,0.00,0.00}{#1}}
\newcommand{\ControlFlowTok}[1]{\textcolor[rgb]{0.13,0.29,0.53}{\textbf{#1}}}
\newcommand{\OperatorTok}[1]{\textcolor[rgb]{0.81,0.36,0.00}{\textbf{#1}}}
\newcommand{\BuiltInTok}[1]{#1}
\newcommand{\ExtensionTok}[1]{#1}
\newcommand{\PreprocessorTok}[1]{\textcolor[rgb]{0.56,0.35,0.01}{\textit{#1}}}
\newcommand{\AttributeTok}[1]{\textcolor[rgb]{0.77,0.63,0.00}{#1}}
\newcommand{\RegionMarkerTok}[1]{#1}
\newcommand{\InformationTok}[1]{\textcolor[rgb]{0.56,0.35,0.01}{\textbf{\textit{#1}}}}
\newcommand{\WarningTok}[1]{\textcolor[rgb]{0.56,0.35,0.01}{\textbf{\textit{#1}}}}
\newcommand{\AlertTok}[1]{\textcolor[rgb]{0.94,0.16,0.16}{#1}}
\newcommand{\ErrorTok}[1]{\textcolor[rgb]{0.64,0.00,0.00}{\textbf{#1}}}
\newcommand{\NormalTok}[1]{#1}
\usepackage{graphicx,grffile}
\makeatletter
\def\maxwidth{\ifdim\Gin@nat@width>\linewidth\linewidth\else\Gin@nat@width\fi}
\def\maxheight{\ifdim\Gin@nat@height>\textheight\textheight\else\Gin@nat@height\fi}
\makeatother
% Scale images if necessary, so that they will not overflow the page
% margins by default, and it is still possible to overwrite the defaults
% using explicit options in \includegraphics[width, height, ...]{}
\setkeys{Gin}{width=\maxwidth,height=\maxheight,keepaspectratio}
\IfFileExists{parskip.sty}{%
\usepackage{parskip}
}{% else
\setlength{\parindent}{0pt}
\setlength{\parskip}{6pt plus 2pt minus 1pt}
}
\setlength{\emergencystretch}{3em}  % prevent overfull lines
\providecommand{\tightlist}{%
  \setlength{\itemsep}{0pt}\setlength{\parskip}{0pt}}
\setcounter{secnumdepth}{0}
% Redefines (sub)paragraphs to behave more like sections
\ifx\paragraph\undefined\else
\let\oldparagraph\paragraph
\renewcommand{\paragraph}[1]{\oldparagraph{#1}\mbox{}}
\fi
\ifx\subparagraph\undefined\else
\let\oldsubparagraph\subparagraph
\renewcommand{\subparagraph}[1]{\oldsubparagraph{#1}\mbox{}}
\fi

%%% Use protect on footnotes to avoid problems with footnotes in titles
\let\rmarkdownfootnote\footnote%
\def\footnote{\protect\rmarkdownfootnote}

%%% Change title format to be more compact
\usepackage{titling}

% Create subtitle command for use in maketitle
\newcommand{\subtitle}[1]{
  \posttitle{
    \begin{center}\large#1\end{center}
    }
}

\setlength{\droptitle}{-2em}
  \title{ChebyChev Analysis}
  \pretitle{\vspace{\droptitle}\centering\huge}
  \posttitle{\par}
  \author{Kory Illenye}
  \preauthor{\centering\large\emph}
  \postauthor{\par}
  \predate{\centering\large\emph}
  \postdate{\par}
  \date{4/19/2018}


\begin{document}
\maketitle

\section{Steepest Descent and Steepest Descent with Least
Squares.}\label{steepest-descent-and-steepest-descent-with-least-squares.}

Lets take a look at two iterative methods for finding solutions to
positive definite, symetric matrices. Steepest Descent is often called
\textbf{gradient descent}, which is a approach to finding minima of a
function \(f\). Using an estimate \(x_i\), the negative gradient
\(-\nabla f(x_i)\) gives the direction of in which \(f\) is decreasing
at the greatest rate. The goal is to take steps in this direction until
we react the minimum.

Notation:

\(e\) denotes an error term

\(i\) denotes the i-ith term

\(r\) denotes a residual

\[e_i = x_i - x\] \[r_i = b - Ax_i = -Ae_i\]

So, the next item to worry about is step size, lets call this
\(\alpha\), and use \(x_{i+1} = x_i - \alpha_i\nabla f(x_i)\) this
minimizes \(f(x_{i+1})\). lets look at some code.

\begin{Shaded}
\begin{Highlighting}[]
\CommentTok{#setting the seed to work with }
\KeywordTok{set.seed}\NormalTok{(}\DecValTok{31}\NormalTok{)}
\CommentTok{#Number of columns and rows in the matrix in the matrix}
\NormalTok{n <-}\StringTok{ }\DecValTok{5} 
\CommentTok{# Creates a Random matrix with n^2 randomly generated numbers}
\NormalTok{maxIts <-}\StringTok{ }\DecValTok{1000000}
\NormalTok{A<-}\KeywordTok{matrix}\NormalTok{(}\KeywordTok{runif}\NormalTok{(n}\OperatorTok{^}\DecValTok{2}\NormalTok{), }\DataTypeTok{ncol=}\NormalTok{n)}
\CommentTok{# Forces the matrix to be symmetric}
\NormalTok{A <-}\StringTok{ }\KeywordTok{t}\NormalTok{(A) }\OperatorTok\StringTok{ }\NormalTok{A}


\CommentTok{# Creates a random solution vector to solve for}
\NormalTok{x <-}\StringTok{ }\KeywordTok{matrix}\NormalTok{(}\KeywordTok{runif}\NormalTok{(n), }\DataTypeTok{ncol =} \DecValTok{1}\NormalTok{)}
\CommentTok{# Creates a b vector from our random vector x}
\NormalTok{b <-}\StringTok{ }\NormalTok{A }\OperatorTok\StringTok{ }\NormalTok{x}
\CommentTok{# initial approximation vector (zero vecor)}
\NormalTok{x0 <-}\StringTok{ }\KeywordTok{matrix}\NormalTok{(}\KeywordTok{rep}\NormalTok{(}\FloatTok{0.0}\NormalTok{,n), }\DataTypeTok{ncol =} \DecValTok{1}\NormalTok{)}
\CommentTok{# Setting a tolerance for the norm of our vector}
\NormalTok{tol =}\StringTok{ }\DecValTok{10}\OperatorTok{^-}\DecValTok{5}
\CommentTok{# determines if all the eigen values are positive}
\KeywordTok{is.positive.definite}\NormalTok{(A)}
\end{Highlighting}
\end{Shaded}

\begin{verbatim}
[1] TRUE
\end{verbatim}

The return of TRUE above allows us to know that the matrix we are using
is positive definite. Meanin that it has all positive eigen values. Now
that all the components have been created for the use in the function.
Lets see what the matrix looks like and design the function.

\subsubsection{The Matrix A}\label{the-matrix-a}

\begin{Shaded}
\begin{Highlighting}[]
\CommentTok{#prints the matrix A}
\NormalTok{A}
\end{Highlighting}
\end{Shaded}

\begin{verbatim}
          [,1]      [,2]      [,3]      [,4]      [,5]
[1,] 2.4299775 1.7269373 2.0199944 0.5590044 1.5099956
[2,] 1.7269373 1.8590925 1.1651029 0.4614977 1.1855963
[3,] 2.0199944 1.1651029 1.8962176 0.5224703 1.2304534
[4,] 0.5590044 0.4614977 0.5224703 0.3137412 0.4209155
[5,] 1.5099956 1.1855963 1.2304534 0.4209155 1.0294046
\end{verbatim}

\subsubsection{The b vector}\label{the-b-vector}

\begin{Shaded}
\begin{Highlighting}[]
\CommentTok{#Prints b}
\NormalTok{b}
\end{Highlighting}
\end{Shaded}

\begin{verbatim}
          [,1]
[1,] 2.2159566
[2,] 1.4658554
[3,] 1.9866331
[4,] 0.6013222
[5,] 1.3916532
\end{verbatim}

\section{The Funtion of Gradient Descent
Method}\label{the-funtion-of-gradient-descent-method}

\begin{Shaded}
\begin{Highlighting}[]
\CommentTok{# defines the function}
\NormalTok{grad <-}\StringTok{ }\ControlFlowTok{function}\NormalTok{ (A,b,x0, maxIts, tol)}
\NormalTok{  \{}
    \CommentTok{# creates an initial residual}
\NormalTok{    r <-}\StringTok{ }\NormalTok{b }\OperatorTok{-}\StringTok{ }\NormalTok{(A }\OperatorTok\StringTok{ }\NormalTok{x0)}
    \CommentTok{# used to keep track of the number of iterations.}
\NormalTok{    k =}\StringTok{ }\DecValTok{0}
    \CommentTok{# loop used to attempt to find the approximation of the X vector}
    \ControlFlowTok{while}\NormalTok{ (k }\OperatorTok{<}\StringTok{ }\NormalTok{maxIts }\OperatorTok{&&}\StringTok{ }\KeywordTok{norm}\NormalTok{(r, }\StringTok{"1"}\NormalTok{) }\OperatorTok{>}\StringTok{ }\NormalTok{tol)}
\NormalTok{    \{}
      \CommentTok{# increments the iteration}
\NormalTok{      k <-}\StringTok{ }\NormalTok{k}\OperatorTok{+}\DecValTok{1}
      \CommentTok{# creates a step size using the residuals and matrix A}
\NormalTok{      alpha <-}\StringTok{ }\NormalTok{(}\KeywordTok{t}\NormalTok{(r) }\OperatorTok\StringTok{ }\NormalTok{r) }\OperatorTok{/}\StringTok{ }\NormalTok{(}\KeywordTok{t}\NormalTok{(r) }\OperatorTok\StringTok{ }\NormalTok{(A }\OperatorTok\StringTok{ }\NormalTok{r))}
      \CommentTok{# creates the next approximation of the X vector}
\NormalTok{      x0 <-}\StringTok{ }\NormalTok{x0 }\OperatorTok{+}\StringTok{ }\KeywordTok{drop}\NormalTok{(alpha) }\OperatorTok{*}\StringTok{ }\NormalTok{r}
      \CommentTok{# creates the new residual with the new approximation}
\NormalTok{      r <-}\StringTok{ }\NormalTok{b }\OperatorTok{-}\StringTok{ }\NormalTok{A }\OperatorTok\StringTok{ }\NormalTok{x0}
\NormalTok{    \}}
  \CommentTok{#Calculates the residual vector.}
\NormalTok{  res <-}\StringTok{ }\NormalTok{x }\OperatorTok{-}\StringTok{ }\NormalTok{x0}
  \CommentTok{# vairable to store the exact X, the aproximated X and the number of iterations}
\NormalTok{  sol <-}\StringTok{ }\KeywordTok{cbind}\NormalTok{(x, x0, res, }\KeywordTok{matrix}\NormalTok{(}\KeywordTok{rep}\NormalTok{(k,n),}\DataTypeTok{ncol =} \DecValTok{1}\NormalTok{))}
  \CommentTok{# Names the colums of the new array}
  \KeywordTok{colnames}\NormalTok{(sol) <-}\StringTok{ }\KeywordTok{c}\NormalTok{(}\StringTok{"X"}\NormalTok{, }\StringTok{"X0"}\NormalTok{,}\StringTok{"resid"}\NormalTok{, }\StringTok{"iter"}\NormalTok{)}
  \CommentTok{# prints the array}
\NormalTok{  sol}
 
\NormalTok{\}}
\end{Highlighting}
\end{Shaded}

This code when run generates the following:

\begin{Shaded}
\begin{Highlighting}[]
\CommentTok{# runs the function and saves it to a variable "c"}
\NormalTok{c <-}\StringTok{ }\KeywordTok{grad}\NormalTok{(A, b, x0, maxIts, tol)}
\CommentTok{# prints the array }
\NormalTok{c }
\end{Highlighting}
\end{Shaded}

\begin{verbatim}
              X         X0         resid iter
[1,] 0.04876597 0.04895561 -1.896346e-04  325
[2,] 0.14752424 0.14746647  5.776931e-05  325
[3,] 0.81504717 0.81491134  1.358297e-04  325
[4,] 0.16098630 0.16109232 -1.060204e-04  325
[5,] 0.07040319 0.07030940  9.378641e-05  325
\end{verbatim}

I set the seed so this data can be replicated. As you can see the
approximation is very close to the actual solution. The norm of the
residual is 10\^{}\{-5\} when rounded to 5 decimal places. For this
document we know that if the \texttt{iter} column is not equal to `r
maxits

\section{6x6 matrix}\label{x6-matrix}

\begin{Shaded}
\begin{Highlighting}[]
\KeywordTok{set.seed}\NormalTok{(}\DecValTok{21}\NormalTok{)}
\NormalTok{n <-}\StringTok{ }\DecValTok{6} 
\CommentTok{# Creates a Random matrix with n^2 randomly generated numbers}
\NormalTok{A<-}\KeywordTok{matrix}\NormalTok{(}\KeywordTok{runif}\NormalTok{(n}\OperatorTok{^}\DecValTok{2}\NormalTok{), }\DataTypeTok{ncol=}\NormalTok{n)}
\CommentTok{# Forces the matrix to be symmetric}
\NormalTok{A <-}\StringTok{ }\KeywordTok{t}\NormalTok{(A) }\OperatorTok\StringTok{ }\NormalTok{A}
\CommentTok{# Creates a random solution vector to solve for}
\NormalTok{x <-}\StringTok{ }\KeywordTok{matrix}\NormalTok{(}\KeywordTok{runif}\NormalTok{(n), }\DataTypeTok{ncol =} \DecValTok{1}\NormalTok{)}
\CommentTok{# Creates a b vector from our random vector x}
\NormalTok{b <-}\StringTok{ }\NormalTok{A }\OperatorTok\StringTok{ }\NormalTok{x}
\CommentTok{# initial approximation vector (zero vecor)}
\NormalTok{x0 <-}\StringTok{ }\KeywordTok{matrix}\NormalTok{(}\KeywordTok{rep}\NormalTok{(}\FloatTok{0.0}\NormalTok{,n), }\DataTypeTok{ncol =} \DecValTok{1}\NormalTok{)}
\CommentTok{# Setting a tolerance for the norm of our vector}
\NormalTok{tol =}\StringTok{ }\DecValTok{10}\OperatorTok{^-}\DecValTok{5}
\CommentTok{# determines if all the eigen values are positive}
\KeywordTok{is.positive.definite}\NormalTok{(A)}
\end{Highlighting}
\end{Shaded}

\begin{verbatim}
[1] TRUE
\end{verbatim}

Lets see how the function does on this 6x6 matrix.

\begin{Shaded}
\begin{Highlighting}[]
\NormalTok{sol<-}\KeywordTok{grad}\NormalTok{(A, b, x0, maxIts, tol)}
\NormalTok{sol}
\end{Highlighting}
\end{Shaded}

\begin{verbatim}
             X        X0         resid iter
[1,] 0.2163319 0.2164881 -0.0001561833 1781
[2,] 0.6504235 0.6502254  0.0001980371 1781
[3,] 0.3351660 0.3348874  0.0002786669 1781
[4,] 0.5076559 0.5075127  0.0001432318 1781
[5,] 0.6528394 0.6532767 -0.0004373487 1781
[6,] 0.9655767 0.9654451  0.0001315399 1781
\end{verbatim}

I want to look at a few matrices, I am going to adjust the seed number
for each. These numbers are arbitrary but for replication purposes I am
setting the seed.

\begin{Shaded}
\begin{Highlighting}[]
\KeywordTok{set.seed}\NormalTok{(}\DecValTok{214}\NormalTok{)}
\NormalTok{n <-}\StringTok{ }\DecValTok{6} 
\CommentTok{# Creates a Random matrix with n^2 randomly generated numbers}
\NormalTok{A<-}\KeywordTok{matrix}\NormalTok{(}\KeywordTok{runif}\NormalTok{(n}\OperatorTok{^}\DecValTok{2}\NormalTok{), }\DataTypeTok{ncol=}\NormalTok{n)}
\CommentTok{# Forces the matrix to be symmetric}
\NormalTok{A <-}\StringTok{ }\KeywordTok{t}\NormalTok{(A) }\OperatorTok\StringTok{ }\NormalTok{A}
\CommentTok{# Creates a random solution vector to solve for}
\NormalTok{x <-}\StringTok{ }\KeywordTok{matrix}\NormalTok{(}\KeywordTok{runif}\NormalTok{(n), }\DataTypeTok{ncol =} \DecValTok{1}\NormalTok{)}
\CommentTok{# Creates a b vector from our random vector x}
\NormalTok{b <-}\StringTok{ }\NormalTok{A }\OperatorTok\StringTok{ }\NormalTok{x}
\CommentTok{# initial approximation vector (zero vecor)}
\NormalTok{x0 <-}\StringTok{ }\KeywordTok{matrix}\NormalTok{(}\KeywordTok{rep}\NormalTok{(}\FloatTok{0.0}\NormalTok{,n), }\DataTypeTok{ncol =} \DecValTok{1}\NormalTok{)}
\CommentTok{# Setting a tolerance for the norm of our vector}
\NormalTok{tol =}\StringTok{ }\DecValTok{10}\OperatorTok{^-}\DecValTok{5}
\CommentTok{# determines if all the eigen values are positive}
\KeywordTok{is.positive.definite}\NormalTok{(A)}
\end{Highlighting}
\end{Shaded}

\begin{verbatim}
[1] TRUE
\end{verbatim}

Lets see how the function does on this 6x6 matrix.

\begin{Shaded}
\begin{Highlighting}[]
\NormalTok{A}
\end{Highlighting}
\end{Shaded}

\begin{verbatim}
         [,1]     [,2]      [,3]      [,4]      [,5]      [,6]
[1,] 2.591707 2.217206 1.1814162 1.5399616 1.6181376 1.1860408
[2,] 2.217206 2.650669 1.6894440 2.0397363 1.4540539 1.1738115
[3,] 1.181416 1.689444 1.3483715 1.3084386 0.9020424 0.5422738
[4,] 1.539962 2.039736 1.3084386 2.0160104 1.2108004 0.8519311
[5,] 1.618138 1.454054 0.9020424 1.2108004 1.4474137 0.4330670
[6,] 1.186041 1.173812 0.5422738 0.8519311 0.4330670 0.9206222
\end{verbatim}

\begin{Shaded}
\begin{Highlighting}[]
\NormalTok{sol <-}\KeywordTok{grad}\NormalTok{(A, b, x0, maxIts, tol)}
\NormalTok{sol}
\end{Highlighting}
\end{Shaded}

\begin{verbatim}
             X        X0         resid iter
[1,] 0.6807363 0.6806130  1.233517e-04 1419
[2,] 0.1331188 0.1331240 -5.250738e-06 1419
[3,] 0.4157805 0.4157955 -1.500242e-05 1419
[4,] 0.8683657 0.8683055  6.016602e-05 1419
[5,] 0.6214230 0.6215572 -1.341803e-04 1419
[6,] 0.1328657 0.1330050 -1.392501e-04 1419
\end{verbatim}

We can see that the function completed after 1419 iterations. This means
that the norm is within \(10^{-5}\).

\begin{Shaded}
\begin{Highlighting}[]
\KeywordTok{set.seed}\NormalTok{(}\DecValTok{2145}\NormalTok{)}
\NormalTok{n <-}\StringTok{ }\DecValTok{6} 
\CommentTok{# Creates a Random matrix with n^2 randomly generated numbers}
\NormalTok{A<-}\KeywordTok{matrix}\NormalTok{(}\KeywordTok{runif}\NormalTok{(n}\OperatorTok{^}\DecValTok{2}\NormalTok{), }\DataTypeTok{ncol=}\NormalTok{n)}
\CommentTok{# Forces the matrix to be symmetric}
\NormalTok{A <-}\StringTok{ }\KeywordTok{t}\NormalTok{(A) }\OperatorTok\StringTok{ }\NormalTok{A}
\CommentTok{# Creates a random solution vector to solve for}
\NormalTok{x <-}\StringTok{ }\KeywordTok{matrix}\NormalTok{(}\KeywordTok{runif}\NormalTok{(n), }\DataTypeTok{ncol =} \DecValTok{1}\NormalTok{)}
\CommentTok{# Creates a b vector from our random vector x}
\NormalTok{b <-}\StringTok{ }\NormalTok{A }\OperatorTok\StringTok{ }\NormalTok{x}
\CommentTok{# initial approximation vector (zero vecor)}
\NormalTok{x0 <-}\StringTok{ }\KeywordTok{matrix}\NormalTok{(}\KeywordTok{rep}\NormalTok{(}\FloatTok{0.0}\NormalTok{,n), }\DataTypeTok{ncol =} \DecValTok{1}\NormalTok{)}
\CommentTok{# Setting a tolerance for the norm of our vector}
\NormalTok{tol =}\StringTok{ }\DecValTok{10}\OperatorTok{^-}\DecValTok{5}
\CommentTok{# determines if all the eigen values are positive}
\KeywordTok{is.positive.definite}\NormalTok{(A)}
\end{Highlighting}
\end{Shaded}

\begin{verbatim}
[1] TRUE
\end{verbatim}

Lets see how the function does on this 6x6 matrix.

\begin{Shaded}
\begin{Highlighting}[]
\NormalTok{A}
\end{Highlighting}
\end{Shaded}

\begin{verbatim}
         [,1]      [,2]      [,3]     [,4]     [,5]     [,6]
[1,] 1.629180 1.3253076 1.3704596 1.897338 1.492470 1.388390
[2,] 1.325308 2.1268329 0.9046964 2.031244 1.432492 1.145834
[3,] 1.370460 0.9046964 1.6388220 1.852486 1.570883 1.469332
[4,] 1.897338 2.0312444 1.8524862 2.844479 2.251959 1.748982
[5,] 1.492470 1.4324924 1.5708827 2.251959 2.753722 1.775626
[6,] 1.388390 1.1458340 1.4693319 1.748982 1.775626 1.692876
\end{verbatim}

\begin{Shaded}
\begin{Highlighting}[]
\NormalTok{sol<-}\StringTok{ }\KeywordTok{grad}\NormalTok{(A, b, x0, maxIts, tol)}
\NormalTok{sol}
\end{Highlighting}
\end{Shaded}

\begin{verbatim}
             X        X0         resid iter
[1,] 0.6258299 0.6258653 -3.544603e-05 1515
[2,] 0.2011419 0.2014055 -2.636294e-04 1515
[3,] 0.5883483 0.5887837 -4.354756e-04 1515
[4,] 0.8560448 0.8556266  4.181889e-04 1515
[5,] 0.9363852 0.9364996 -1.144587e-04 1515
[6,] 0.9780356 0.9777614  2.741660e-04 1515
\end{verbatim}

We can see that the function completed after 1515 iterations. This means
that the norm is within \(10^{-5}\).

\begin{Shaded}
\begin{Highlighting}[]
\KeywordTok{set.seed}\NormalTok{(}\DecValTok{21456}\NormalTok{)}
\NormalTok{n <-}\StringTok{ }\DecValTok{6} 
\CommentTok{# Creates a Random matrix with n^2 randomly generated numbers}
\NormalTok{A<-}\KeywordTok{matrix}\NormalTok{(}\KeywordTok{runif}\NormalTok{(n}\OperatorTok{^}\DecValTok{2}\NormalTok{), }\DataTypeTok{ncol=}\NormalTok{n)}
\CommentTok{# Forces the matrix to be symmetric}
\NormalTok{A <-}\StringTok{ }\KeywordTok{t}\NormalTok{(A) }\OperatorTok\StringTok{ }\NormalTok{A}
\CommentTok{# Creates a random solution vector to solve for}
\NormalTok{x <-}\StringTok{ }\KeywordTok{matrix}\NormalTok{(}\KeywordTok{runif}\NormalTok{(n), }\DataTypeTok{ncol =} \DecValTok{1}\NormalTok{)}
\CommentTok{# Creates a b vector from our random vector x}
\NormalTok{b <-}\StringTok{ }\NormalTok{A }\OperatorTok\StringTok{ }\NormalTok{x}
\CommentTok{# initial approximation vector (zero vecor)}
\NormalTok{x0 <-}\StringTok{ }\KeywordTok{matrix}\NormalTok{(}\KeywordTok{rep}\NormalTok{(}\FloatTok{0.0}\NormalTok{,n), }\DataTypeTok{ncol =} \DecValTok{1}\NormalTok{)}
\CommentTok{# Setting a tolerance for the norm of our vector}
\NormalTok{tol =}\StringTok{ }\DecValTok{10}\OperatorTok{^-}\DecValTok{5}
\CommentTok{# determines if all the eigen values are positive}
\KeywordTok{is.positive.definite}\NormalTok{(A)}
\end{Highlighting}
\end{Shaded}

\begin{verbatim}
[1] TRUE
\end{verbatim}

Lets see how the function does on this 6x6 matrix.

\begin{Shaded}
\begin{Highlighting}[]
\NormalTok{A}
\end{Highlighting}
\end{Shaded}

\begin{verbatim}
          [,1]      [,2]     [,3]      [,4]      [,5]      [,6]
[1,] 2.5322214 0.9122529 2.004203 1.5482151 0.7751318 2.0283032
[2,] 0.9122529 2.0519766 2.134736 0.9660084 0.7108958 1.5908699
[3,] 2.0042025 2.1347358 3.714131 1.4712356 1.1478843 2.0035704
[4,] 1.5482151 0.9660084 1.471236 1.2711860 0.6324079 1.4763406
[5,] 0.7751318 0.7108958 1.147884 0.6324079 0.4117123 0.7575121
[6,] 2.0283032 1.5908699 2.003570 1.4763406 0.7575121 2.3369386
\end{verbatim}

\begin{Shaded}
\begin{Highlighting}[]
\NormalTok{sol<-}\StringTok{ }\KeywordTok{grad}\NormalTok{(A, b, x0, maxIts, tol)}
\NormalTok{sol}
\end{Highlighting}
\end{Shaded}

\begin{verbatim}
              X         X0         resid iter
[1,] 0.79899443 0.79902045 -2.601344e-05 2127
[2,] 0.43135773 0.43143786 -8.012212e-05 2127
[3,] 0.07193685 0.07209572 -1.588676e-04 2127
[4,] 0.10669362 0.10699923 -3.056067e-04 2127
[5,] 0.33782973 0.33693283  8.969094e-04 2127
[6,] 0.15723681 0.15712123  1.155732e-04 2127
\end{verbatim}

We can see that the function completed after 2127 iterations. This means
that the norm is within \(10^{-5}\).

\section{10x10 matrix}\label{x10-matrix}

\begin{Shaded}
\begin{Highlighting}[]
\KeywordTok{set.seed}\NormalTok{(}\DecValTok{21456}\NormalTok{)}
\NormalTok{n <-}\StringTok{ }\DecValTok{10} 
\CommentTok{# Creates a Random matrix with n^2 randomly generated numbers}
\NormalTok{A<-}\KeywordTok{matrix}\NormalTok{(}\KeywordTok{runif}\NormalTok{(n}\OperatorTok{^}\DecValTok{2}\NormalTok{), }\DataTypeTok{ncol=}\NormalTok{n)}
\CommentTok{# Forces the matrix to be symmetric}
\NormalTok{A <-}\StringTok{ }\KeywordTok{t}\NormalTok{(A) }\OperatorTok\StringTok{ }\NormalTok{A}
\CommentTok{# Creates a random solution vector to solve for}
\NormalTok{x <-}\StringTok{ }\KeywordTok{matrix}\NormalTok{(}\KeywordTok{runif}\NormalTok{(n), }\DataTypeTok{ncol =} \DecValTok{1}\NormalTok{)}
\CommentTok{# Creates a b vector from our random vector x}
\NormalTok{b <-}\StringTok{ }\NormalTok{A }\OperatorTok\StringTok{ }\NormalTok{x}
\CommentTok{# initial approximation vector (zero vecor)}
\NormalTok{x0 <-}\StringTok{ }\KeywordTok{matrix}\NormalTok{(}\KeywordTok{rep}\NormalTok{(}\FloatTok{0.0}\NormalTok{,n), }\DataTypeTok{ncol =} \DecValTok{1}\NormalTok{)}
\CommentTok{# Setting a tolerance for the norm of our vector}
\NormalTok{tol =}\StringTok{ }\DecValTok{10}\OperatorTok{^-}\DecValTok{5}
\CommentTok{# determines if all the eigen values are positive}
\KeywordTok{is.positive.definite}\NormalTok{(A)}
\end{Highlighting}
\end{Shaded}

\begin{verbatim}
[1] TRUE
\end{verbatim}

\begin{Shaded}
\begin{Highlighting}[]
\NormalTok{A}
\end{Highlighting}
\end{Shaded}

\begin{verbatim}
          [,1]     [,2]      [,3]     [,4]     [,5]      [,6]     [,7]
 [1,] 4.576307 3.544017 1.8706107 2.699453 3.464121 2.1412326 3.630429
 [2,] 3.544017 3.995759 1.7402252 2.908020 3.122860 1.7878213 3.701816
 [3,] 1.870611 1.740225 1.4091612 1.632820 1.585560 0.8950613 1.883202
 [4,] 2.699453 2.908020 1.6328204 3.177959 2.352454 1.6335524 2.982156
 [5,] 3.464121 3.122860 1.5855600 2.352454 3.832469 2.1929775 3.852252
 [6,] 2.141233 1.787821 0.8950613 1.633552 2.192978 2.0398368 2.512283
 [7,] 3.630429 3.701816 1.8832023 2.982156 3.852252 2.5122828 5.087222
 [8,] 2.783155 2.232120 1.2940313 1.823810 2.891021 1.8346436 3.536088
 [9,] 2.300886 2.567662 1.2002483 2.224262 2.623019 1.8329427 3.223934
[10,] 2.345922 2.103801 1.4733256 1.801321 2.664459 1.7837057 2.810596
          [,8]     [,9]    [,10]
 [1,] 2.783155 2.300886 2.345922
 [2,] 2.232120 2.567662 2.103801
 [3,] 1.294031 1.200248 1.473326
 [4,] 1.823810 2.224262 1.801321
 [5,] 2.891021 2.623019 2.664459
 [6,] 1.834644 1.832943 1.783706
 [7,] 3.536088 3.223934 2.810596
 [8,] 3.133619 2.283068 1.976192
 [9,] 2.283068 2.591424 1.922565
[10,] 1.976192 1.922565 2.373093
\end{verbatim}

\begin{Shaded}
\begin{Highlighting}[]
\NormalTok{sol<-}\StringTok{ }\KeywordTok{grad}\NormalTok{(A, b, x0, maxIts, tol)}
\NormalTok{sol}
\end{Highlighting}
\end{Shaded}

\begin{verbatim}
              X        X0         resid iter
 [1,] 0.6172931 0.6173013 -8.226232e-06 3532
 [2,] 0.8323071 0.8323258 -1.870730e-05 3532
 [3,] 0.5397727 0.5397052  6.745871e-05 3532
 [4,] 0.1785781 0.1785965 -1.844572e-05 3532
 [5,] 0.7888260 0.7887850  4.097310e-05 3532
 [6,] 0.3950113 0.3949807  3.059918e-05 3532
 [7,] 0.7050482 0.7050369  1.135881e-05 3532
 [8,] 0.8651250 0.8651531 -2.812421e-05 3532
 [9,] 0.1558820 0.1558699  1.212071e-05 3532
[10,] 0.3583763 0.3584488 -7.249651e-05 3532
\end{verbatim}

We can see that the function completed after 3532 iterations. This means
that the norm is within \(10^{-5}\).

\section{Gradient Descent with least
squares}\label{gradient-descent-with-least-squares}

This function is supposed to converge faster.

\begin{Shaded}
\begin{Highlighting}[]
\CommentTok{# defines the function}
\NormalTok{gradLS <-}\StringTok{ }\ControlFlowTok{function}\NormalTok{ (A,b,x0, maxIts, tol)}
\NormalTok{  \{}
    \CommentTok{# creates an initial residual}
\NormalTok{    r <-}\StringTok{ }\NormalTok{b }\OperatorTok{-}\StringTok{ }\NormalTok{(A }\OperatorTok\StringTok{ }\NormalTok{x0)}
    \CommentTok{# used to keep track of the number of iterations.}
\NormalTok{    k =}\StringTok{ }\DecValTok{0}
    \CommentTok{# loop used to attempt to find the approximation of the X vector}
    \ControlFlowTok{while}\NormalTok{ (k }\OperatorTok{<}\StringTok{ }\NormalTok{maxIts }\OperatorTok{&&}\StringTok{ }\KeywordTok{norm}\NormalTok{(r, }\StringTok{"1"}\NormalTok{) }\OperatorTok{>}\StringTok{ }\NormalTok{tol)}
\NormalTok{    \{}
      \CommentTok{# increments the iteration}
\NormalTok{      k <-}\StringTok{ }\NormalTok{k}\OperatorTok{+}\DecValTok{1}
      \CommentTok{# creates a step size using the residuals and matrix A}
\NormalTok{      alpha <-}\StringTok{ }\NormalTok{(}\KeywordTok{t}\NormalTok{(r) }\OperatorTok\StringTok{ }\NormalTok{r) }\OperatorTok{/}\StringTok{ }\NormalTok{(}\KeywordTok{t}\NormalTok{(A}\OperatorTok\NormalTok{r) }\OperatorTok\StringTok{ }\NormalTok{(A }\OperatorTok\StringTok{ }\NormalTok{r))}
      \CommentTok{# creates the next approximation of the X vector}
\NormalTok{      x0 <-}\StringTok{ }\NormalTok{x0 }\OperatorTok{+}\StringTok{ }\KeywordTok{drop}\NormalTok{(alpha) }\OperatorTok{*}\StringTok{ }\NormalTok{r}
      \CommentTok{# creates the new residual with the new approximation}
\NormalTok{      r <-}\StringTok{ }\KeywordTok{t}\NormalTok{(A)}\OperatorTok\StringTok{ }\NormalTok{b }\OperatorTok{-}\StringTok{ }\NormalTok{(}\KeywordTok{t}\NormalTok{(A) }\OperatorTok\StringTok{ }\NormalTok{A) }\OperatorTok\StringTok{ }\NormalTok{x0}
\NormalTok{    \}}
  \CommentTok{#Calculates the residual vector.}
\NormalTok{  res <-}\StringTok{ }\NormalTok{x }\OperatorTok{-}\StringTok{ }\NormalTok{x0}
  \CommentTok{# vairable to store the exact X, the aproximated X and the number of iterations}
\NormalTok{  sol <-}\StringTok{ }\KeywordTok{cbind}\NormalTok{(x, x0, res, }\KeywordTok{matrix}\NormalTok{(}\KeywordTok{rep}\NormalTok{(k,n),}\DataTypeTok{ncol =} \DecValTok{1}\NormalTok{))}
  \CommentTok{# Names the colums of the new array}
  \KeywordTok{colnames}\NormalTok{(sol) <-}\StringTok{ }\KeywordTok{c}\NormalTok{(}\StringTok{"X"}\NormalTok{, }\StringTok{"X0"}\NormalTok{,}\StringTok{"resid"}\NormalTok{, }\StringTok{"iter"}\NormalTok{)}
  \CommentTok{# prints the array}
\NormalTok{  sol}
 
\NormalTok{\}}
\end{Highlighting}
\end{Shaded}

\begin{Shaded}
\begin{Highlighting}[]
\NormalTok{A}
\end{Highlighting}
\end{Shaded}

\begin{verbatim}
          [,1]     [,2]      [,3]     [,4]     [,5]      [,6]     [,7]
 [1,] 4.576307 3.544017 1.8706107 2.699453 3.464121 2.1412326 3.630429
 [2,] 3.544017 3.995759 1.7402252 2.908020 3.122860 1.7878213 3.701816
 [3,] 1.870611 1.740225 1.4091612 1.632820 1.585560 0.8950613 1.883202
 [4,] 2.699453 2.908020 1.6328204 3.177959 2.352454 1.6335524 2.982156
 [5,] 3.464121 3.122860 1.5855600 2.352454 3.832469 2.1929775 3.852252
 [6,] 2.141233 1.787821 0.8950613 1.633552 2.192978 2.0398368 2.512283
 [7,] 3.630429 3.701816 1.8832023 2.982156 3.852252 2.5122828 5.087222
 [8,] 2.783155 2.232120 1.2940313 1.823810 2.891021 1.8346436 3.536088
 [9,] 2.300886 2.567662 1.2002483 2.224262 2.623019 1.8329427 3.223934
[10,] 2.345922 2.103801 1.4733256 1.801321 2.664459 1.7837057 2.810596
          [,8]     [,9]    [,10]
 [1,] 2.783155 2.300886 2.345922
 [2,] 2.232120 2.567662 2.103801
 [3,] 1.294031 1.200248 1.473326
 [4,] 1.823810 2.224262 1.801321
 [5,] 2.891021 2.623019 2.664459
 [6,] 1.834644 1.832943 1.783706
 [7,] 3.536088 3.223934 2.810596
 [8,] 3.133619 2.283068 1.976192
 [9,] 2.283068 2.591424 1.922565
[10,] 1.976192 1.922565 2.373093
\end{verbatim}

\begin{Shaded}
\begin{Highlighting}[]
\NormalTok{sol<-}\StringTok{ }\KeywordTok{gradLS}\NormalTok{(A, b, x0, maxIts, tol)}
\NormalTok{sol}
\end{Highlighting}
\end{Shaded}

\begin{verbatim}
              X        X0         resid   iter
 [1,] 0.6172931 0.6177202 -0.0004271701 914619
 [2,] 0.8323071 0.8332761 -0.0009690351 914619
 [3,] 0.5397727 0.5362854  0.0034873060 914619
 [4,] 0.1785781 0.1795333 -0.0009552130 914619
 [5,] 0.7888260 0.7867091  0.0021169184 914619
 [6,] 0.3950113 0.3934302  0.0015811167 914619
 [7,] 0.7050482 0.7044629  0.0005853126 914619
 [8,] 0.8651250 0.8665807 -0.0014557269 914619
 [9,] 0.1558820 0.1552567  0.0006253598 914619
[10,] 0.3583763 0.3621263 -0.0037499883 914619
\end{verbatim}

Here is the point my mind went, What!!!! This took
9.14619\times 10\^{}\{5\} iterations. Why? I had to know more. initially
I decided lets mess with the code a little.

So on the line that creates the new residual I removed the additional
transposes of A.

\begin{Shaded}
\begin{Highlighting}[]
\CommentTok{# defines the function}
\NormalTok{gradLS2 <-}\StringTok{ }\ControlFlowTok{function}\NormalTok{ (A,b,x0, maxIts, tol)}
\NormalTok{  \{}
    \CommentTok{# creates an initial residual}
\NormalTok{    r <-}\StringTok{ }\NormalTok{b }\OperatorTok{-}\StringTok{ }\NormalTok{(A }\OperatorTok\StringTok{ }\NormalTok{x0)}
    \CommentTok{# used to keep track of the number of iterations.}
\NormalTok{    k =}\StringTok{ }\DecValTok{0}
    \CommentTok{# loop used to attempt to find the approximation of the X vector}
    \ControlFlowTok{while}\NormalTok{ (k }\OperatorTok{<}\StringTok{ }\NormalTok{maxIts }\OperatorTok{&&}\StringTok{ }\KeywordTok{norm}\NormalTok{(r, }\StringTok{"1"}\NormalTok{) }\OperatorTok{>}\StringTok{ }\NormalTok{tol)}
\NormalTok{    \{}
      \CommentTok{# increments the iteration}
\NormalTok{      k <-}\StringTok{ }\NormalTok{k}\OperatorTok{+}\DecValTok{1}
      \CommentTok{# creates a step size using the residuals and matrix A}
\NormalTok{      alpha <-}\StringTok{ }\NormalTok{(}\KeywordTok{t}\NormalTok{(r) }\OperatorTok\StringTok{ }\NormalTok{r) }\OperatorTok{/}\StringTok{ }\NormalTok{(}\KeywordTok{t}\NormalTok{(A}\OperatorTok\NormalTok{r) }\OperatorTok\StringTok{ }\NormalTok{(A }\OperatorTok\StringTok{ }\NormalTok{r))}
      \CommentTok{# creates the next approximation of the X vector}
\NormalTok{      x0 <-}\StringTok{ }\NormalTok{x0 }\OperatorTok{+}\StringTok{ }\KeywordTok{drop}\NormalTok{(alpha) }\OperatorTok{*}\StringTok{ }\NormalTok{r}
      \CommentTok{# creates the new residual with the new approximation}
\NormalTok{      r <-}\StringTok{  }\NormalTok{b }\OperatorTok{-}\StringTok{ }\NormalTok{(A) }\OperatorTok\StringTok{ }\NormalTok{x0}
\NormalTok{    \}}
  \CommentTok{#Calculates the residual vector.}
\NormalTok{  res <-}\StringTok{ }\NormalTok{x }\OperatorTok{-}\StringTok{ }\NormalTok{x0}
  \CommentTok{# vairable to store the exact X, the aproximated X and the number of iterations}
\NormalTok{  sol <-}\StringTok{ }\KeywordTok{cbind}\NormalTok{(x, x0, res, }\KeywordTok{matrix}\NormalTok{(}\KeywordTok{rep}\NormalTok{(k,n),}\DataTypeTok{ncol =} \DecValTok{1}\NormalTok{))}
  \CommentTok{# Names the colums of the new array}
  \KeywordTok{colnames}\NormalTok{(sol) <-}\StringTok{ }\KeywordTok{c}\NormalTok{(}\StringTok{"X"}\NormalTok{, }\StringTok{"X0"}\NormalTok{,}\StringTok{"resid"}\NormalTok{, }\StringTok{"iter"}\NormalTok{)}
  \CommentTok{# prints the array}
\NormalTok{  sol}
 
\NormalTok{\}}
\end{Highlighting}
\end{Shaded}

Same matrix but slightly different function

\begin{Shaded}
\begin{Highlighting}[]
\NormalTok{A}
\end{Highlighting}
\end{Shaded}

\begin{verbatim}
          [,1]     [,2]      [,3]     [,4]     [,5]      [,6]     [,7]
 [1,] 4.576307 3.544017 1.8706107 2.699453 3.464121 2.1412326 3.630429
 [2,] 3.544017 3.995759 1.7402252 2.908020 3.122860 1.7878213 3.701816
 [3,] 1.870611 1.740225 1.4091612 1.632820 1.585560 0.8950613 1.883202
 [4,] 2.699453 2.908020 1.6328204 3.177959 2.352454 1.6335524 2.982156
 [5,] 3.464121 3.122860 1.5855600 2.352454 3.832469 2.1929775 3.852252
 [6,] 2.141233 1.787821 0.8950613 1.633552 2.192978 2.0398368 2.512283
 [7,] 3.630429 3.701816 1.8832023 2.982156 3.852252 2.5122828 5.087222
 [8,] 2.783155 2.232120 1.2940313 1.823810 2.891021 1.8346436 3.536088
 [9,] 2.300886 2.567662 1.2002483 2.224262 2.623019 1.8329427 3.223934
[10,] 2.345922 2.103801 1.4733256 1.801321 2.664459 1.7837057 2.810596
          [,8]     [,9]    [,10]
 [1,] 2.783155 2.300886 2.345922
 [2,] 2.232120 2.567662 2.103801
 [3,] 1.294031 1.200248 1.473326
 [4,] 1.823810 2.224262 1.801321
 [5,] 2.891021 2.623019 2.664459
 [6,] 1.834644 1.832943 1.783706
 [7,] 3.536088 3.223934 2.810596
 [8,] 3.133619 2.283068 1.976192
 [9,] 2.283068 2.591424 1.922565
[10,] 1.976192 1.922565 2.373093
\end{verbatim}

\begin{Shaded}
\begin{Highlighting}[]
\NormalTok{sol<-}\StringTok{ }\KeywordTok{gradLS2}\NormalTok{(A, b, x0, maxIts, tol)}
\NormalTok{sol}
\end{Highlighting}
\end{Shaded}

\begin{verbatim}
              X        X0         resid iter
 [1,] 0.6172931 0.6173075 -1.444088e-05  886
 [2,] 0.8323071 0.8322895  1.756001e-05  886
 [3,] 0.5397727 0.5397693  3.335324e-06  886
 [4,] 0.1785781 0.1785764  1.682705e-06  886
 [5,] 0.7888260 0.7888242  1.814743e-06  886
 [6,] 0.3950113 0.3949988  1.251022e-05  886
 [7,] 0.7050482 0.7050616 -1.332729e-05  886
 [8,] 0.8651250 0.8651073  1.765960e-05  886
 [9,] 0.1558820 0.1558986 -1.654262e-05  886
[10,] 0.3583763 0.3583780 -1.650070e-06  886
\end{verbatim}

Well, thats wierd 886 iterations. This seems interesting we went from
3532 to 914619 to 886 iterations. Well this fixed the problem for this
matrix but does it work for all?

\begin{Shaded}
\begin{Highlighting}[]
\KeywordTok{set.seed}\NormalTok{(}\DecValTok{214567}\NormalTok{)}
\NormalTok{n <-}\StringTok{ }\DecValTok{5}
\CommentTok{# Creates a Random matrix with n^2 randomly generated numbers}
\NormalTok{A<-}\KeywordTok{matrix}\NormalTok{(}\KeywordTok{runif}\NormalTok{(n}\OperatorTok{^}\DecValTok{2}\NormalTok{), }\DataTypeTok{ncol=}\NormalTok{n)}
\CommentTok{# Forces the matrix to be symmetric}
\NormalTok{A <-}\StringTok{ }\KeywordTok{t}\NormalTok{(A) }\OperatorTok\StringTok{ }\NormalTok{A}
\CommentTok{# Creates a random solution vector to solve for}
\NormalTok{x <-}\StringTok{ }\KeywordTok{matrix}\NormalTok{(}\KeywordTok{runif}\NormalTok{(n), }\DataTypeTok{ncol =} \DecValTok{1}\NormalTok{)}
\CommentTok{# Creates a b vector from our random vector x}
\NormalTok{b <-}\StringTok{ }\NormalTok{A }\OperatorTok\StringTok{ }\NormalTok{x}
\CommentTok{# initial approximation vector (zero vecor)}
\NormalTok{x0 <-}\StringTok{ }\KeywordTok{matrix}\NormalTok{(}\KeywordTok{rep}\NormalTok{(}\FloatTok{0.0}\NormalTok{,n), }\DataTypeTok{ncol =} \DecValTok{1}\NormalTok{)}
\CommentTok{# Setting a tolerance for the norm of our vector}
\NormalTok{tol =}\StringTok{ }\DecValTok{10}\OperatorTok{^-}\DecValTok{5}
\CommentTok{# determines if all the eigen values are positive}
\KeywordTok{is.positive.definite}\NormalTok{(A)}
\end{Highlighting}
\end{Shaded}

\begin{verbatim}
[1] TRUE
\end{verbatim}

\begin{Shaded}
\begin{Highlighting}[]
\NormalTok{A}
\end{Highlighting}
\end{Shaded}

\begin{verbatim}
          [,1]     [,2]      [,3]      [,4]      [,5]
[1,] 1.3951977 1.439403 0.9458712 0.6875470 0.3530277
[2,] 1.4394026 2.826384 2.1103633 1.3952379 1.3081469
[3,] 0.9458712 2.110363 1.8090746 0.9263186 1.1690460
[4,] 0.6875470 1.395238 0.9263186 0.8325923 0.6920785
[5,] 0.3530277 1.308147 1.1690460 0.6920785 1.0185438
\end{verbatim}

\begin{Shaded}
\begin{Highlighting}[]
\NormalTok{sol1<-}\StringTok{ }\KeywordTok{grad}\NormalTok{(A, b, x0, maxIts, tol)}
\NormalTok{sol2<-}\StringTok{ }\KeywordTok{gradLS}\NormalTok{(A, b, x0, maxIts, tol)}
\NormalTok{sol3<-}\StringTok{ }\KeywordTok{gradLS2}\NormalTok{(A, b, x0, maxIts, tol)}
\NormalTok{sol1}
\end{Highlighting}
\end{Shaded}

\begin{verbatim}
              X         X0         resid iter
[1,] 0.28642738 0.28633495  9.243224e-05 4667
[2,] 0.09199288 0.09073066  1.262219e-03 4667
[3,] 0.36928209 0.37065175 -1.369664e-03 4667
[4,] 0.03981780 0.04120865 -1.390849e-03 4667
[5,] 0.95860400 0.95773918  8.648254e-04 4667
\end{verbatim}

\begin{Shaded}
\begin{Highlighting}[]
\NormalTok{sol2}
\end{Highlighting}
\end{Shaded}

\begin{verbatim}
              X          X0        resid iter
[1,] 0.28642738 0.280137231  0.006290151  486
[2,] 0.09199288 0.008120223  0.083872660  486
[3,] 0.36928209 0.460459605 -0.091177516  486
[4,] 0.03981780 0.132354278 -0.092536480  486
[5,] 0.95860400 0.900839980  0.057764024  486
\end{verbatim}

\begin{Shaded}
\begin{Highlighting}[]
\NormalTok{sol3}
\end{Highlighting}
\end{Shaded}

\begin{verbatim}
              X         X0         resid iter
[1,] 0.28642738 0.28639605  3.133395e-05 1439
[2,] 0.09199288 0.09156893  4.239535e-04 1439
[3,] 0.36928209 0.36974119 -4.591008e-04 1439
[4,] 0.03981780 0.04028413 -4.663322e-04 1439
[5,] 0.95860400 0.95831367  2.903382e-04 1439
\end{verbatim}

Well this matrix broke my theory again. Back to the drawing board. I
decided maybe I just had a bad psuedo code. I decided to look into
another iterative method and found the advanced chevy cheb iterative
method that avoids dot products of residuals.

\section{Advanced Cheby Chev iterative
method}\label{advanced-cheby-chev-iterative-method}

So I did some research and found some psuedo code to play with.

\begin{Shaded}
\begin{Highlighting}[]
\NormalTok{cheby <-}\StringTok{ }\ControlFlowTok{function}\NormalTok{ (A,b,x0, maxIts, tol)}
\NormalTok{\{}
\NormalTok{  ev <-}\StringTok{ }\KeywordTok{eigen}\NormalTok{(A)}
\NormalTok{  lmax<-}\StringTok{ }\KeywordTok{max}\NormalTok{(ev}\OperatorTok{$}\NormalTok{values)}
\NormalTok{  lmin <-}\StringTok{ }\KeywordTok{min}\NormalTok{(ev}\OperatorTok{$}\NormalTok{values)}
  
\NormalTok{  d<-}\StringTok{ }\NormalTok{(lmax }\OperatorTok{+}\NormalTok{lmin)}\OperatorTok{/}\FloatTok{2.0}
\NormalTok{  c<-}\StringTok{ }\NormalTok{(lmax}\OperatorTok{-}\NormalTok{lmin)}\OperatorTok{/}\FloatTok{2.0}
\NormalTok{  precon <-}\StringTok{ }\KeywordTok{diag}\NormalTok{(}\KeywordTok{length}\NormalTok{(A[,}\DecValTok{1}\NormalTok{]))}
\NormalTok{  x <-}\StringTok{ }\NormalTok{x0}
\NormalTok{  r <-}\StringTok{ }\NormalTok{b }\OperatorTok{-}\StringTok{ }\NormalTok{A }\OperatorTok\StringTok{ }\NormalTok{x}
\NormalTok{  i <-}\StringTok{ }\DecValTok{1}
  \ControlFlowTok{while}\NormalTok{(i }\OperatorTok{<}\StringTok{ }\NormalTok{maxIts }\OperatorTok{&&}\StringTok{ }\KeywordTok{norm}\NormalTok{(r, }\StringTok{"1"}\NormalTok{) }\OperatorTok{>}\StringTok{ }\NormalTok{tol)}
\NormalTok{  \{}
\NormalTok{    z =}\StringTok{ }\KeywordTok{solve}\NormalTok{(precon, r)}
    \ControlFlowTok{if}\NormalTok{ (i }\OperatorTok{==}\StringTok{ }\DecValTok{1}\NormalTok{)}
\NormalTok{    \{}
\NormalTok{      p <-}\StringTok{ }\NormalTok{z}
\NormalTok{      alpha <-}\StringTok{ }\DecValTok{1}\OperatorTok{/}\NormalTok{d}
\NormalTok{    \}}
    \ControlFlowTok{else}
\NormalTok{    \{}
\NormalTok{      beta <-}\StringTok{ }\NormalTok{(c}\OperatorTok{*}\NormalTok{alpha}\OperatorTok{/}\FloatTok{2.0}\NormalTok{)}\OperatorTok{^}\DecValTok{2}
\NormalTok{      alpha <-}\StringTok{ }\DecValTok{1}\OperatorTok{/}\NormalTok{(d}\OperatorTok{-}\NormalTok{beta}\OperatorTok{/}\NormalTok{alpha)}
\NormalTok{      p <-}\StringTok{ }\NormalTok{z }\OperatorTok{+}\StringTok{ }\NormalTok{beta}\OperatorTok{*}\NormalTok{p}
\NormalTok{    \}}
\NormalTok{    x <-}\StringTok{ }\NormalTok{x }\OperatorTok{+}\StringTok{ }\NormalTok{alpha}\OperatorTok{*}\NormalTok{p}
\NormalTok{    r <-}\StringTok{ }\NormalTok{b }\OperatorTok{-}\StringTok{ }\NormalTok{A }\OperatorTok\StringTok{ }\NormalTok{x}
\NormalTok{    i =}\StringTok{ }\NormalTok{i}\OperatorTok{+}\DecValTok{1}
\NormalTok{  \}}
  \CommentTok{#Calculates the residual vector.}
\NormalTok{  res <-}\StringTok{ }\NormalTok{x }\OperatorTok{-}\StringTok{ }\NormalTok{x0}
  \CommentTok{# vairable to store the exact X, the aproximated X and the number of iterations}
\NormalTok{  sol <-}\StringTok{ }\KeywordTok{cbind}\NormalTok{(x, x0, res, }\KeywordTok{matrix}\NormalTok{(}\KeywordTok{rep}\NormalTok{(i}\OperatorTok{-}\DecValTok{1}\NormalTok{,n),}\DataTypeTok{ncol =} \DecValTok{1}\NormalTok{))}
  \CommentTok{# Names the colums of the new array}
  \KeywordTok{colnames}\NormalTok{(sol) <-}\StringTok{ }\KeywordTok{c}\NormalTok{(}\StringTok{"X"}\NormalTok{, }\StringTok{"X0"}\NormalTok{,}\StringTok{"resid"}\NormalTok{, }\StringTok{"iter"}\NormalTok{)}
  \CommentTok{# prints the array}
\NormalTok{  sol}
\NormalTok{\}}
\end{Highlighting}
\end{Shaded}

lets see how it does?

\begin{Shaded}
\begin{Highlighting}[]
\NormalTok{A}
\end{Highlighting}
\end{Shaded}

\begin{verbatim}
          [,1]     [,2]      [,3]      [,4]      [,5]
[1,] 1.3951977 1.439403 0.9458712 0.6875470 0.3530277
[2,] 1.4394026 2.826384 2.1103633 1.3952379 1.3081469
[3,] 0.9458712 2.110363 1.8090746 0.9263186 1.1690460
[4,] 0.6875470 1.395238 0.9263186 0.8325923 0.6920785
[5,] 0.3530277 1.308147 1.1690460 0.6920785 1.0185438
\end{verbatim}

\begin{Shaded}
\begin{Highlighting}[]
\NormalTok{sol1<-}\StringTok{ }\KeywordTok{grad}\NormalTok{(A, b, x0, maxIts, tol)}
\NormalTok{sol2<-}\StringTok{ }\KeywordTok{gradLS}\NormalTok{(A, b, x0, maxIts, tol)}
\NormalTok{sol3<-}\StringTok{ }\KeywordTok{gradLS2}\NormalTok{(A, b, x0, maxIts, tol)}
\NormalTok{sol4<-}\StringTok{ }\KeywordTok{cheby}\NormalTok{(A, b, x0, maxIts, tol)}
\NormalTok{sol1}
\end{Highlighting}
\end{Shaded}

\begin{verbatim}
              X         X0         resid iter
[1,] 0.28642738 0.28633495  9.243224e-05 4667
[2,] 0.09199288 0.09073066  1.262219e-03 4667
[3,] 0.36928209 0.37065175 -1.369664e-03 4667
[4,] 0.03981780 0.04120865 -1.390849e-03 4667
[5,] 0.95860400 0.95773918  8.648254e-04 4667
\end{verbatim}

\begin{Shaded}
\begin{Highlighting}[]
\NormalTok{sol2}
\end{Highlighting}
\end{Shaded}

\begin{verbatim}
              X          X0        resid iter
[1,] 0.28642738 0.280137231  0.006290151  486
[2,] 0.09199288 0.008120223  0.083872660  486
[3,] 0.36928209 0.460459605 -0.091177516  486
[4,] 0.03981780 0.132354278 -0.092536480  486
[5,] 0.95860400 0.900839980  0.057764024  486
\end{verbatim}

\begin{Shaded}
\begin{Highlighting}[]
\NormalTok{sol3}
\end{Highlighting}
\end{Shaded}

\begin{verbatim}
              X         X0         resid iter
[1,] 0.28642738 0.28639605  3.133395e-05 1439
[2,] 0.09199288 0.09156893  4.239535e-04 1439
[3,] 0.36928209 0.36974119 -4.591008e-04 1439
[4,] 0.03981780 0.04028413 -4.663322e-04 1439
[5,] 0.95860400 0.95831367  2.903382e-04 1439
\end{verbatim}

\begin{Shaded}
\begin{Highlighting}[]
\NormalTok{sol4}
\end{Highlighting}
\end{Shaded}

\begin{verbatim}
              X X0      resid iter
[1,] 0.28642714  0 0.28642714  496
[2,] 0.09199234  0 0.09199234  496
[3,] 0.36928183  0 0.36928183  496
[4,] 0.03981767  0 0.03981767  496
[5,] 0.95860372  0 0.95860372  496
\end{verbatim}

Interesting this matrix it only took 496 iterations. Lets try a few more
from earlier.

\begin{Shaded}
\begin{Highlighting}[]
\KeywordTok{set.seed}\NormalTok{(}\DecValTok{21456}\NormalTok{)}
\NormalTok{n <-}\StringTok{ }\DecValTok{10} 
\CommentTok{# Creates a Random matrix with n^2 randomly generated numbers}
\NormalTok{A<-}\KeywordTok{matrix}\NormalTok{(}\KeywordTok{runif}\NormalTok{(n}\OperatorTok{^}\DecValTok{2}\NormalTok{), }\DataTypeTok{ncol=}\NormalTok{n)}
\CommentTok{# Forces the matrix to be symmetric}
\NormalTok{A <-}\StringTok{ }\KeywordTok{t}\NormalTok{(A) }\OperatorTok\StringTok{ }\NormalTok{A}
\CommentTok{# Creates a random solution vector to solve for}
\NormalTok{x <-}\StringTok{ }\KeywordTok{matrix}\NormalTok{(}\KeywordTok{runif}\NormalTok{(n), }\DataTypeTok{ncol =} \DecValTok{1}\NormalTok{)}
\CommentTok{# Creates a b vector from our random vector x}
\NormalTok{b <-}\StringTok{ }\NormalTok{A }\OperatorTok\StringTok{ }\NormalTok{x}
\CommentTok{# initial approximation vector (zero vecor)}
\NormalTok{x0 <-}\StringTok{ }\KeywordTok{matrix}\NormalTok{(}\KeywordTok{rep}\NormalTok{(}\FloatTok{0.0}\NormalTok{,n), }\DataTypeTok{ncol =} \DecValTok{1}\NormalTok{)}
\CommentTok{# Setting a tolerance for the norm of our vector}
\NormalTok{tol =}\StringTok{ }\DecValTok{10}\OperatorTok{^-}\DecValTok{5}
\CommentTok{# determines if all the eigen values are positive}
\KeywordTok{is.positive.definite}\NormalTok{(A)}
\end{Highlighting}
\end{Shaded}

\begin{verbatim}
[1] TRUE
\end{verbatim}

\begin{Shaded}
\begin{Highlighting}[]
\NormalTok{A}
\end{Highlighting}
\end{Shaded}

\begin{verbatim}
          [,1]     [,2]      [,3]     [,4]     [,5]      [,6]     [,7]
 [1,] 4.576307 3.544017 1.8706107 2.699453 3.464121 2.1412326 3.630429
 [2,] 3.544017 3.995759 1.7402252 2.908020 3.122860 1.7878213 3.701816
 [3,] 1.870611 1.740225 1.4091612 1.632820 1.585560 0.8950613 1.883202
 [4,] 2.699453 2.908020 1.6328204 3.177959 2.352454 1.6335524 2.982156
 [5,] 3.464121 3.122860 1.5855600 2.352454 3.832469 2.1929775 3.852252
 [6,] 2.141233 1.787821 0.8950613 1.633552 2.192978 2.0398368 2.512283
 [7,] 3.630429 3.701816 1.8832023 2.982156 3.852252 2.5122828 5.087222
 [8,] 2.783155 2.232120 1.2940313 1.823810 2.891021 1.8346436 3.536088
 [9,] 2.300886 2.567662 1.2002483 2.224262 2.623019 1.8329427 3.223934
[10,] 2.345922 2.103801 1.4733256 1.801321 2.664459 1.7837057 2.810596
          [,8]     [,9]    [,10]
 [1,] 2.783155 2.300886 2.345922
 [2,] 2.232120 2.567662 2.103801
 [3,] 1.294031 1.200248 1.473326
 [4,] 1.823810 2.224262 1.801321
 [5,] 2.891021 2.623019 2.664459
 [6,] 1.834644 1.832943 1.783706
 [7,] 3.536088 3.223934 2.810596
 [8,] 3.133619 2.283068 1.976192
 [9,] 2.283068 2.591424 1.922565
[10,] 1.976192 1.922565 2.373093
\end{verbatim}

\begin{Shaded}
\begin{Highlighting}[]
\NormalTok{sol<-}\StringTok{ }\KeywordTok{cheby}\NormalTok{(A, b, x0, maxIts, tol)}
\NormalTok{sol}
\end{Highlighting}
\end{Shaded}

\begin{verbatim}
              X X0     resid iter
 [1,] 0.6172931  0 0.6172931  325
 [2,] 0.8323071  0 0.8323071  325
 [3,] 0.5397727  0 0.5397727  325
 [4,] 0.1785781  0 0.1785781  325
 [5,] 0.7888260  0 0.7888260  325
 [6,] 0.3950113  0 0.3950113  325
 [7,] 0.7050483  0 0.7050483  325
 [8,] 0.8651250  0 0.8651250  325
 [9,] 0.1558820  0 0.1558820  325
[10,] 0.3583763  0 0.3583763  325
\end{verbatim}

wow, on the 10x10 it only took 325 iterations!

\begin{Shaded}
\begin{Highlighting}[]
\KeywordTok{set.seed}\NormalTok{(}\DecValTok{21456}\NormalTok{)}
\NormalTok{n <-}\StringTok{ }\DecValTok{6} 
\CommentTok{# Creates a Random matrix with n^2 randomly generated numbers}
\NormalTok{A<-}\KeywordTok{matrix}\NormalTok{(}\KeywordTok{runif}\NormalTok{(n}\OperatorTok{^}\DecValTok{2}\NormalTok{), }\DataTypeTok{ncol=}\NormalTok{n)}
\CommentTok{# Forces the matrix to be symmetric}
\NormalTok{A <-}\StringTok{ }\KeywordTok{t}\NormalTok{(A) }\OperatorTok\StringTok{ }\NormalTok{A}
\CommentTok{# Creates a random solution vector to solve for}
\NormalTok{x <-}\StringTok{ }\KeywordTok{matrix}\NormalTok{(}\KeywordTok{runif}\NormalTok{(n), }\DataTypeTok{ncol =} \DecValTok{1}\NormalTok{)}
\CommentTok{# Creates a b vector from our random vector x}
\NormalTok{b <-}\StringTok{ }\NormalTok{A }\OperatorTok\StringTok{ }\NormalTok{x}
\CommentTok{# initial approximation vector (zero vecor)}
\NormalTok{x0 <-}\StringTok{ }\KeywordTok{matrix}\NormalTok{(}\KeywordTok{rep}\NormalTok{(}\FloatTok{0.0}\NormalTok{,n), }\DataTypeTok{ncol =} \DecValTok{1}\NormalTok{)}
\CommentTok{# Setting a tolerance for the norm of our vector}
\NormalTok{tol =}\StringTok{ }\DecValTok{10}\OperatorTok{^-}\DecValTok{5}
\CommentTok{# determines if all the eigen values are positive}
\KeywordTok{is.positive.definite}\NormalTok{(A)}
\end{Highlighting}
\end{Shaded}

\begin{verbatim}
[1] TRUE
\end{verbatim}

Lets see how the function does on this 6x6 matrix.

\begin{Shaded}
\begin{Highlighting}[]
\NormalTok{A}
\end{Highlighting}
\end{Shaded}

\begin{verbatim}
          [,1]      [,2]     [,3]      [,4]      [,5]      [,6]
[1,] 2.5322214 0.9122529 2.004203 1.5482151 0.7751318 2.0283032
[2,] 0.9122529 2.0519766 2.134736 0.9660084 0.7108958 1.5908699
[3,] 2.0042025 2.1347358 3.714131 1.4712356 1.1478843 2.0035704
[4,] 1.5482151 0.9660084 1.471236 1.2711860 0.6324079 1.4763406
[5,] 0.7751318 0.7108958 1.147884 0.6324079 0.4117123 0.7575121
[6,] 2.0283032 1.5908699 2.003570 1.4763406 0.7575121 2.3369386
\end{verbatim}

\begin{Shaded}
\begin{Highlighting}[]
\NormalTok{sol<-}\StringTok{ }\KeywordTok{cheby}\NormalTok{(A, b, x0, maxIts, tol)}
\NormalTok{sol}
\end{Highlighting}
\end{Shaded}

\begin{verbatim}
              X X0      resid iter
[1,] 0.79899462  0 0.79899462  393
[2,] 0.43135791  0 0.43135791  393
[3,] 0.07193712  0 0.07193712  393
[4,] 0.10669381  0 0.10669381  393
[5,] 0.33782969  0 0.33782969  393
[6,] 0.15723699  0 0.15723699  393
\end{verbatim}

Even better for this 6x6! Well what is different between gradient
descent and cheby chevy. Well Cheby Chevy iteration avoids the dot
products and ratio of residuals. It uses an eliptical appraoch to
solving the problem. This means we have to have some knowledge of the
eigen values of our matrix.

After talking to Dr.~Palmer I realized that part of the problem might be
that the matrices are ill conditioned. So, lets redfine a few matrices
and look at the condition numbers.

\begin{Shaded}
\begin{Highlighting}[]
\KeywordTok{set.seed}\NormalTok{(}\DecValTok{21456}\NormalTok{)}
\NormalTok{n <-}\StringTok{ }\DecValTok{10} 
\CommentTok{# Creates a Random matrix with n^2 randomly generated numbers}
\NormalTok{A<-}\KeywordTok{matrix}\NormalTok{(}\KeywordTok{runif}\NormalTok{(n}\OperatorTok{^}\DecValTok{2}\NormalTok{), }\DataTypeTok{ncol=}\NormalTok{n)}
\CommentTok{# Forces the matrix to be symmetric}
\NormalTok{A <-}\StringTok{ }\KeywordTok{t}\NormalTok{(A) }\OperatorTok\StringTok{ }\NormalTok{A}
\CommentTok{# Creates a random solution vector to solve for}
\NormalTok{x <-}\StringTok{ }\KeywordTok{matrix}\NormalTok{(}\KeywordTok{runif}\NormalTok{(n), }\DataTypeTok{ncol =} \DecValTok{1}\NormalTok{)}
\CommentTok{# Creates a b vector from our random vector x}
\NormalTok{b <-}\StringTok{ }\NormalTok{A }\OperatorTok\StringTok{ }\NormalTok{x}
\CommentTok{# initial approximation vector (zero vecor)}
\NormalTok{x0 <-}\StringTok{ }\KeywordTok{matrix}\NormalTok{(}\KeywordTok{rep}\NormalTok{(}\FloatTok{0.0}\NormalTok{,n), }\DataTypeTok{ncol =} \DecValTok{1}\NormalTok{)}
\CommentTok{# Setting a tolerance for the norm of our vector}
\NormalTok{tol =}\StringTok{ }\DecValTok{10}\OperatorTok{^-}\DecValTok{5}
\CommentTok{# determines if all the eigen values are positive}
\KeywordTok{is.positive.definite}\NormalTok{(A)}
\end{Highlighting}
\end{Shaded}

\begin{verbatim}
[1] TRUE
\end{verbatim}

\begin{Shaded}
\begin{Highlighting}[]
\KeywordTok{set.seed}\NormalTok{(}\DecValTok{2145}\NormalTok{)}
\NormalTok{n <-}\StringTok{ }\DecValTok{6} 
\CommentTok{# Creates a Random matrix with n^2 randomly generated numbers}
\NormalTok{A<-}\KeywordTok{matrix}\NormalTok{(}\KeywordTok{runif}\NormalTok{(n}\OperatorTok{^}\DecValTok{2}\NormalTok{), }\DataTypeTok{ncol=}\NormalTok{n)}
\CommentTok{# Forces the matrix to be symmetric}
\NormalTok{B <-}\StringTok{ }\KeywordTok{t}\NormalTok{(A) }\OperatorTok\StringTok{ }\NormalTok{A}
\CommentTok{# Creates a random solution vector to solve for}
\NormalTok{x <-}\StringTok{ }\KeywordTok{matrix}\NormalTok{(}\KeywordTok{runif}\NormalTok{(n), }\DataTypeTok{ncol =} \DecValTok{1}\NormalTok{)}
\CommentTok{# Creates a b vector from our random vector x}
\NormalTok{b <-}\StringTok{ }\NormalTok{B }\OperatorTok\StringTok{ }\NormalTok{x}
\CommentTok{# initial approximation vector (zero vecor)}
\NormalTok{x0 <-}\StringTok{ }\KeywordTok{matrix}\NormalTok{(}\KeywordTok{rep}\NormalTok{(}\FloatTok{0.0}\NormalTok{,n), }\DataTypeTok{ncol =} \DecValTok{1}\NormalTok{)}
\CommentTok{# Setting a tolerance for the norm of our vector}
\NormalTok{tol =}\StringTok{ }\DecValTok{10}\OperatorTok{^-}\DecValTok{5}
\CommentTok{# determines if all the eigen values are positive}
\KeywordTok{is.positive.definite}\NormalTok{(B)}
\end{Highlighting}
\end{Shaded}

\begin{verbatim}
[1] TRUE
\end{verbatim}

\begin{Shaded}
\begin{Highlighting}[]
\KeywordTok{set.seed}\NormalTok{(}\DecValTok{214}\NormalTok{)}
\NormalTok{n <-}\StringTok{ }\DecValTok{6} 
\CommentTok{# Creates a Random matrix with n^2 randomly generated numbers}
\NormalTok{A<-}\KeywordTok{matrix}\NormalTok{(}\KeywordTok{runif}\NormalTok{(n}\OperatorTok{^}\DecValTok{2}\NormalTok{), }\DataTypeTok{ncol=}\NormalTok{n)}
\CommentTok{# Forces the matrix to be symmetric}
\NormalTok{C <-}\StringTok{ }\KeywordTok{t}\NormalTok{(A) }\OperatorTok\StringTok{ }\NormalTok{A}
\CommentTok{# Creates a random solution vector to solve for}
\NormalTok{x <-}\StringTok{ }\KeywordTok{matrix}\NormalTok{(}\KeywordTok{runif}\NormalTok{(n), }\DataTypeTok{ncol =} \DecValTok{1}\NormalTok{)}
\CommentTok{# Creates a b vector from our random vector x}
\NormalTok{b <-}\StringTok{ }\NormalTok{C }\OperatorTok\StringTok{ }\NormalTok{x}
\CommentTok{# initial approximation vector (zero vecor)}
\NormalTok{x0 <-}\StringTok{ }\KeywordTok{matrix}\NormalTok{(}\KeywordTok{rep}\NormalTok{(}\FloatTok{0.0}\NormalTok{,n), }\DataTypeTok{ncol =} \DecValTok{1}\NormalTok{)}
\CommentTok{# Setting a tolerance for the norm of our vector}
\NormalTok{tol =}\StringTok{ }\DecValTok{10}\OperatorTok{^-}\DecValTok{5}
\CommentTok{# determines if all the eigen values are positive}
\KeywordTok{is.positive.definite}\NormalTok{(C)}
\end{Highlighting}
\end{Shaded}

\begin{verbatim}
[1] TRUE
\end{verbatim}

\begin{Shaded}
\begin{Highlighting}[]
\KeywordTok{set.seed}\NormalTok{(}\DecValTok{21}\NormalTok{)}
\NormalTok{n <-}\StringTok{ }\DecValTok{6} 
\CommentTok{# Creates a Random matrix with n^2 randomly generated numbers}
\NormalTok{A<-}\KeywordTok{matrix}\NormalTok{(}\KeywordTok{runif}\NormalTok{(n}\OperatorTok{^}\DecValTok{2}\NormalTok{), }\DataTypeTok{ncol=}\NormalTok{n)}
\CommentTok{# Forces the matrix to be symmetric}
\NormalTok{D <-}\StringTok{ }\KeywordTok{t}\NormalTok{(A) }\OperatorTok\StringTok{ }\NormalTok{A}
\CommentTok{# Creates a random solution vector to solve for}
\NormalTok{x <-}\StringTok{ }\KeywordTok{matrix}\NormalTok{(}\KeywordTok{runif}\NormalTok{(n), }\DataTypeTok{ncol =} \DecValTok{1}\NormalTok{)}
\CommentTok{# Creates a b vector from our random vector x}
\NormalTok{b <-}\StringTok{ }\NormalTok{D }\OperatorTok\StringTok{ }\NormalTok{x}
\CommentTok{# initial approximation vector (zero vecor)}
\NormalTok{x0 <-}\StringTok{ }\KeywordTok{matrix}\NormalTok{(}\KeywordTok{rep}\NormalTok{(}\FloatTok{0.0}\NormalTok{,n), }\DataTypeTok{ncol =} \DecValTok{1}\NormalTok{)}
\CommentTok{# Setting a tolerance for the norm of our vector}
\NormalTok{tol =}\StringTok{ }\DecValTok{10}\OperatorTok{^-}\DecValTok{5}
\CommentTok{# determines if all the eigen values are positive}
\KeywordTok{is.positive.definite}\NormalTok{(D)}
\end{Highlighting}
\end{Shaded}

\begin{verbatim}
[1] TRUE
\end{verbatim}

\begin{Shaded}
\begin{Highlighting}[]
\KeywordTok{kappa}\NormalTok{(A)}
\end{Highlighting}
\end{Shaded}

\begin{verbatim}
[1] 49.10724
\end{verbatim}

\begin{Shaded}
\begin{Highlighting}[]
\KeywordTok{kappa}\NormalTok{(B)}
\end{Highlighting}
\end{Shaded}

\begin{verbatim}
[1] 2148.086
\end{verbatim}

\begin{Shaded}
\begin{Highlighting}[]
\KeywordTok{kappa}\NormalTok{(C)}
\end{Highlighting}
\end{Shaded}

\begin{verbatim}
[1] 434.2473
\end{verbatim}

\begin{Shaded}
\begin{Highlighting}[]
\KeywordTok{kappa}\NormalTok{(D)}
\end{Highlighting}
\end{Shaded}

\begin{verbatim}
[1] 1706.385
\end{verbatim}

Well higher condition numbers mean that the matrices are ill
conditioned. We can see that matrix B is highly ill conditioned. Well
lets condition a matrix and see how that affects the number of
iterations. We can use the B matrix.

\begin{Shaded}
\begin{Highlighting}[]
\KeywordTok{set.seed}\NormalTok{(}\DecValTok{2145}\NormalTok{)}
\NormalTok{n <-}\StringTok{ }\DecValTok{6} 
\CommentTok{# Creates a Random matrix with n^2 randomly generated numbers}
\NormalTok{A<-}\KeywordTok{matrix}\NormalTok{(}\KeywordTok{runif}\NormalTok{(n}\OperatorTok{^}\DecValTok{2}\NormalTok{), }\DataTypeTok{ncol=}\NormalTok{n)}
\CommentTok{# Forces the matrix to be symmetric}
\NormalTok{B <-}\StringTok{ }\NormalTok{(}\KeywordTok{t}\NormalTok{(A) }\OperatorTok\StringTok{ }\NormalTok{A)}
\NormalTok{B<-}\StringTok{ }\NormalTok{B }\OperatorTok{+}\StringTok{ }\NormalTok{(}\DecValTok{100}\OperatorTok{*}\KeywordTok{diag}\NormalTok{(n))}
\CommentTok{# Creates a random solution vector to solve for}
\NormalTok{x <-}\StringTok{ }\KeywordTok{matrix}\NormalTok{(}\KeywordTok{runif}\NormalTok{(n), }\DataTypeTok{ncol =} \DecValTok{1}\NormalTok{)}
\CommentTok{# Creates a b vector from our random vector x}
\NormalTok{b <-}\StringTok{ }\NormalTok{B }\OperatorTok\StringTok{ }\NormalTok{x}
\CommentTok{# initial approximation vector (zero vecor)}
\NormalTok{x0 <-}\StringTok{ }\KeywordTok{matrix}\NormalTok{(}\KeywordTok{rep}\NormalTok{(}\FloatTok{0.0}\NormalTok{,n), }\DataTypeTok{ncol =} \DecValTok{1}\NormalTok{)}
\CommentTok{# Setting a tolerance for the norm of our vector}
\NormalTok{tol =}\StringTok{ }\DecValTok{10}\OperatorTok{^-}\DecValTok{5}
\CommentTok{# determines if all the eigen values are positive}
\KeywordTok{is.positive.definite}\NormalTok{(B)}
\end{Highlighting}
\end{Shaded}

\begin{verbatim}
[1] TRUE
\end{verbatim}

\begin{Shaded}
\begin{Highlighting}[]
\NormalTok{B}
\end{Highlighting}
\end{Shaded}

\begin{verbatim}
           [,1]        [,2]        [,3]       [,4]       [,5]       [,6]
[1,] 101.629180   1.3253076   1.3704596   1.897338   1.492470   1.388390
[2,]   1.325308 102.1268329   0.9046964   2.031244   1.432492   1.145834
[3,]   1.370460   0.9046964 101.6388220   1.852486   1.570883   1.469332
[4,]   1.897338   2.0312444   1.8524862 102.844479   2.251959   1.748982
[5,]   1.492470   1.4324924   1.5708827   2.251959 102.753722   1.775626
[6,]   1.388390   1.1458340   1.4693319   1.748982   1.775626 101.692876
\end{verbatim}

\begin{Shaded}
\begin{Highlighting}[]
\NormalTok{sol1<-}\StringTok{ }\KeywordTok{grad}\NormalTok{(B, b, x0, maxIts, tol)}
\NormalTok{sol2<-}\StringTok{ }\KeywordTok{gradLS}\NormalTok{(B, b, x0, maxIts, tol)}
\NormalTok{sol3<-}\StringTok{ }\KeywordTok{gradLS2}\NormalTok{(B, b, x0, maxIts, tol)}
\NormalTok{sol4<-}\StringTok{ }\KeywordTok{cheby}\NormalTok{(B, b, x0, maxIts, tol)}
\NormalTok{sol1}
\end{Highlighting}
\end{Shaded}

\begin{verbatim}
             X        X0         resid iter
[1,] 0.6258299 0.6258299 -3.726958e-09    5
[2,] 0.2011419 0.2011419 -1.359563e-08    5
[3,] 0.5883483 0.5883483 -2.515808e-08    5
[4,] 0.8560448 0.8560448  6.611190e-09    5
[5,] 0.9363852 0.9363852 -1.016239e-08    5
[6,] 0.9780356 0.9780356  1.431781e-08    5
\end{verbatim}

\begin{Shaded}
\begin{Highlighting}[]
\NormalTok{sol2}
\end{Highlighting}
\end{Shaded}

\begin{verbatim}
             X        X0        resid iter
[1,] 0.6258299 0.6258299 6.393130e-11    9
[2,] 0.2011419 0.2011419 3.926945e-11    9
[3,] 0.5883483 0.5883483 1.334466e-11    9
[4,] 0.8560448 0.8560448 1.281852e-10    9
[5,] 0.9363852 0.9363852 6.868373e-11    9
[6,] 0.9780356 0.9780356 1.035089e-10    9
\end{verbatim}

\begin{Shaded}
\begin{Highlighting}[]
\NormalTok{sol3}
\end{Highlighting}
\end{Shaded}

\begin{verbatim}
             X        X0         resid iter
[1,] 0.6258299 0.6258299  8.458740e-09 1880
[2,] 0.2011419 0.2011419 -1.867218e-08 1880
[3,] 0.5883483 0.5883482  4.345919e-09 1880
[4,] 0.8560448 0.8560448  1.170221e-08 1880
[5,] 0.9363852 0.9363851  1.918453e-08 1880
[6,] 0.9780356 0.9780356  3.216321e-08 1880
\end{verbatim}

\begin{Shaded}
\begin{Highlighting}[]
\NormalTok{sol4}
\end{Highlighting}
\end{Shaded}

\begin{verbatim}
             X X0     resid iter
[1,] 0.6258299  0 0.6258299    6
[2,] 0.2011419  0 0.2011419    6
[3,] 0.5883483  0 0.5883483    6
[4,] 0.8560448  0 0.8560448    6
[5,] 0.9363852  0 0.9363852    6
[6,] 0.9780356  0 0.9780356    6
\end{verbatim}

By conditioning the matrix prior to solving using the iterative method
we can see that it makes a significant differance in solving for the
solution but Cheby Chev's method is still clost to the fastest method
for solving these symetric matrices. There is also a functional form
that can be used for non-symetric matrices. I want to consider using
this on image processing and seeing where this takes me.


\end{document}
