\begin{frame}

Statistical Inference and Hypothesis Testing for Population Proportion
by Kory Illenye

\end{frame}

\begin{frame}{Statistical Inference}

Statistical inference is the process in which we make conclusions of a
population based on sample data.

\end{frame}

\begin{frame}{Where is Statistical Inference used?}

\begin{itemize}
\tightlist
\item
  Manufacturing
\item
  Politics
\item
  Education
\item
  Court Room
\end{itemize}

\end{frame}

\begin{frame}{R vs.~Sally Clark}

\begin{itemize}
\tightlist
\item
  1999 Sally Clark was convicted of mudering her two sons.
\item
  Throughout the case many medical experts had contradictory evidence
\item
  Resounding statistic - Professor Sir Roy Meadow states 1 in 73 million
  chance of this occuring
\item
  2003 Sally was release due to addition evidence that was kept from
  trial
\end{itemize}

\end{frame}

\begin{frame}{Key Criteria for Hypothesis Testing for a Population
Proportion}

\begin{itemize}
\tightlist
\item
  Random: Data must come from a random sample, experiment or simulation
\item
  Normal: The sampling distribution of \(\hat{p}\) needs to be
  relatively normal.

  \begin{itemize}
  \tightlist
  \item
    \(n\) represents sample size and \(p\) represents the population's
    proportion of successes.
  \item
    \(np \geq 10\)
  \item
    \(n(1-p) \geq 10\)
  \end{itemize}
\item
  Independence: If the sample size is less than 10\% of population size
  then we can treat each observation as independent since removing each
  observation doesn't significantly change the population as it is
  sampled.
\end{itemize}

\end{frame}

\begin{frame}{5 Step Process for Testing a Hypothesis of Population
Proportion}

Step 1: State the null hypothesis (\(H_0\)) and the alternate hypothesis
(\(H_a\)).

Step 2: Identify the test statistic (observered proportion)

Step 3: Calculate the Rejection Region (\(\alpha\)-value)

Step 4: State the statistical conclusion

Step 5: State the English conclusion

\end{frame}

\begin{frame}{5 Step Process Example}

A report claimed that 20\% of all college graduates find a job in their
chosen field of study within one year of graduation. A random sample of
500 graduates found that 90 obtained work in there field within one year
of graduation. On a significance level of 0.05 (\(\alpha = 0.05\)) is
there statistical evidence to refute the claim of this report?

\begin{itemize}
\tightlist
\item
  Step 1:

  \begin{itemize}
  \tightlist
  \item
    \(H_0 = 0.20\)
  \item
    \(H_a \neq 0.20\)
  \end{itemize}
\item
  Step 2: \(\hat{p} = \frac{90}{500} = 0.18\)
\end{itemize}

\end{frame}

\begin{frame}{5 Step Process Example (Continued)}

\begin{itemize}
\tightlist
\item
  Step 3: Compare \(p\)-value to \(\alpha = 0.05\). The rejection region
  is when \(p\)-value is less than 0.05.

  \begin{itemize}
  \tightlist
  \item
    sd(\(\hat{p}\)) =
    \(\sqrt{\frac{pq}{n}} = \sqrt{\frac{0.2*0.8}{500}} = 0.01789\)
  \item
    \(z = \frac{\hat{p}-p}{sd(\hat{p})} = \frac{0.18-.2}{0.01789} = -1.12\)
  \item
    \(p\)-value = 2*P(z \textless{} -1.12) = 0.2628
  \item
    Verify
    \href{https://mathcracker.com/z-test-for-one-proportion.php}{proportion
    Calculator}
  \end{itemize}
\item
  Step 4: Since the \(p\)-value of 0.2628 is greater than the \(\alpha\)
  value of 0.05 we fail to reject \(H_0\)
\item
  Step 5: There is not significant enough evidence to suggest the
  proportion of college graduates finding work in there chosen field is
  something other than 20\%.
\end{itemize}

\end{frame}

\begin{frame}{Questions}

\begin{itemize}
\tightlist
\item
  Questions?
\end{itemize}

\end{frame}

\begin{frame}{References}

De Veaux; Velleman; and Bock; \emph{Stats: Data and Models}, Pearson
Education, 2016.

Ugarte, Maria D.; Militino, Ana F.; Arnholt, Alan T.; \emph{Probability
and Statistics with R, Second Edition}, CRC Press, 2016.

Networked Knowledge - Law Report, \emph{R. v. Sally Clark {[}2003{]}
EWCA Crim 1020 {[}Part Two{]}},
\url{http://netk.net.au/UK/SallyClark2.asp}

Z-test: One Pop. Proportion,
\href{https://mathcracker.com/z-test-for-one-proportion.php}{proportion
Calculator}

\end{frame}
